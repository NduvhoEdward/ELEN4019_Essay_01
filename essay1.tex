%%%%%%%%%%%%%%%%%%%%%%%%%%%%%%%%%%%%%%%%%%%%%%%%%%%%%%%%%%%%%%%%%%%%%%%%%%%%%%%
%
%		 witseiepaper-2005.tex
%
%  Nduvho Ramashia - 30 March 2023
%
%  Sociology Essay 1
%
%%%%%%%%%%%%%%%%%%%%%%%%%%%%%%%%%%%%%%%%%%%%%%%%%%%%%%%%%%%%%%%%%%%%%%%%%%%%%%%

\documentclass[12pt]{witseiepaper} 

%
% All KJN's macros and goodies (some shameless borrowing from SPL)
\usepackage{KJN}
\usepackage{comment}
%
% PDF Info
%
\ifpdf
\pdfinfo{
/Title (The Engineer's potential doom)
/Author (Nduvho E Ramashia)
/CreationDate (D:202303302200)
/ModDate (D:202303302200)
/Subject (ELEN4017 Sociology, 2023)/Keywords (ELEN4017, Sociology)
}
\fi

%%%%%%%%%%%%%%%%%%%%%%%%%%%%%%%%%%%%%%%%%%%%%%%%%%%%%%%%%%%%%%%%%%%%%%%%%%%%%%%
\begin{document}

\title{The Looming Era of Ethical Minefields: Challenges awaiting engineers in the decade ahead}

\author{Nduvho E. Ramashia -- 1490804 
\thanks{School of Electrical \& Information Engineering, University of theWitwatersrand, Private Bag 3, 2050, Johannesburg, South Africa} 
} 

%%%%%%%%%%%%%%%%%%%%%%%%%%%%%%%%%%%%%%%%%%%%%%%%%%%%%%%%%%%%%%%%%%%%%%%%%%%%%%%%%

\abstract{N/A}
\keywords{ELEN4017, Sociology}


\maketitle
\thispagestyle{empty}
\pagestyle{empty}

%%%%%%%%%%%%%%%%%%%%%%%%%%%%%%%%%%%%%%%%%%%%%%%%%%%%%%%%%%%%%%%%%%%%%%%%%%%%%%%%

\section{INTRODUCTION} 

    \subsection{The Global Perspective} 

    From its inception, the engineering profession has always been about meeting the ever expanding needs and wants of man \cite{cruickshank2003changing}. The previous centuries engineers brought about many vital and foundational inventions. Most of which we still use to this day \cite{cruickshank2003changing}. This include but not limited inventions such as bridges, cars, electricity and computers, to mention some. However, they never had to worry much about the amount of natural resources they were using. Neither did they have to put much consideration into the overall long-term effects their work were having on the environment, at least not as much as the engineers of the this day have to. Given the abundance of resources they had at their disposal then, it is no surprise. Considering how well and rich the environment was then, it is no surprise why they did not have to worry as much as the engineers of this day and the coming decade will have to worry about the overall impact their designs will be having on the environment. They could not then feel the urgency to protect the environmental, not as much the engineers in the coming decade will. All they had to worry about was about the meeting their engineering duties, fulfilling the needs and wants of people, without having to always consciously worry about whether or not it was moral to implement their solutions or make any design to wanted. On the other hand, these are the challenges the engineers of today are dealing with. They are the difficulties they are having to navigate. They have the need to be more conscious of the effects their work has on the overall ecosystem, while also delivering what the people need. They are having to work with and at an already at-risk environment \cite{macnaghten2006risk}. The Intergovernmental Panel on Climate Change (IPCC) projections derived from simulations performed for the IPCC 4th Assessment Report shows how the rainfalls, amongst other things, will have worsened, as in, decreased by the year 2030 \cite{suppiah2007australian}. These facts and predictions shines upon the pressures the future engineers will have to deal with. They show the difficulties they will have in fulfilling their duties, meeting the needs and wants of man, while also not violating everyone’s environment; while also making sure not to damage even more, significantly, the very then fragile environment. They will be more frequently faced with situations where they will have to provide for the needs of people, while also not harming the environment on the very man lives in. This will be in the form of not finishing off the finite non-renewable resources, and also making sure their products will not, at any stage, have a significant negative impact on the overall ecosystem. On one side they will have the need to fulfil the needs of the people, to ensure decent lives for all. On the other hand they will have to carefully make their design and choices to make sure they do not harm the ecosystem or environment. These facts show the moral challenges the future engineers will face with balancing Edward B. Barbier’s three cycles \cite{barbier1987concept}, more commonly labelled ‘economic’, ‘society’ and ‘environment’. The imbalance of the cycles will results in the violation on the ethical values autonomy, beneficence, and justice. This could be in the form of people not getting or affording descent lives, or not having their health and other needs met. It could also be in the people’s place of residence, the environment, not getting treated well, which is not good the people as a whole. The engineers in a decade’s time will be more frequently faced with situations where they will have to consciously make sure they do not violate any of the general ethical values/rights. 

%--------------------------------------------------------------------------

	\subsection{The Impact on South Africa} 

	Given the fact that South Africa is still a developing country \cite{bakari2018south}, its engineers will be even highly disadvantaged. Other, though not all, parts of the world have made significant technological advances back when the world was less conscious of the environmental and ecosystem damages done by the products engineers made. They as a result got ahead technologically as well as economically. South Africa on the other due to its more pressing issues then, did not. Unfortunately the ecosystem harm was still done for all even then. This puts South Africa and its engineers at a disadvantage since they, now and will still have in to a decade’s time, to be as careful as everyone else. Its engineers will have the same limits every other engineer has on the globe, with no regard to facts such as, who has contributed how much to the then current environment and ecosystem harm. This will present an even more challenging moral challenge to the South African engineers when it comes to providing for the people whilst respecting the environment and the global well- being, and at the same time trying provide the people’s needs, as well as catching up to other countries. While other countries developed with no so tight limitations with regard to the environment and the ecosystem, South Africa will not much of that luxury. Its engineers will have be more careful what natural resources they use to provide for the different needs and wants of the country. 

%%%%%%%%%%%%%%%%%%%%%%%%%%%%%%%%%%%%%%%%%%%%%%%%%%%%%%%%%%%%%%%%%%%%%%%%%%%%%%%%

\section{Problem analysis}

Given the state the environment has been projected to be going to be in the future \cite{collins2014long},thinking about the overall environmental impacts the engineers make will be of even greater importance. 

	\subsection{Deontological View of the Challenge} 

	From the deontological point of view, the natural-resources-using engineers will have to battle with whether or not it will moral or not to solve the challenges they will then be facing. Given the fact that natural resources are finite, hence are running out \cite{subramanian2018crisis}, the future engineers will have to be more careful how much they each use. They will have to be sure they are not wrong in using however much resources they need. More specifically looking at it from the Categorical Imperative point of view, they will have to stay within the bounds of natural resources uses as well as unwanted by-products, as it will be the right thing to do, as per the Categorical Imperative concept \cite{basara2018kant}, since everyone not doing so will ensure the critical damage to the ecosystem \cite{appannagari2017environmental}. On the other hand, they will still have to provide for the needs of the people, as well as to grow and develop the country, for South Africa’s case. They will have to, on every project or innovation, battle with the morality of the project due to the limited resources and the ever dying ecosystem, since they are one of the main professions that work with and use with the environment and whose product contribute to the bad health of the ecosystem  
 %-------------------------------------------------------------------------- 

	\subsection{Teleological Point of View} 

	Looking at it from the teleological point of view, which focuses more on the consequences of an action \cite{ismail2018narrative}, they will still have to be careful, or else they end up bringing about the irreversible damage of the ecosystem, which is not a wanted or acceptable outcome or consequence. Looking at it more specifically from the Utilitarian point of view, whose approach is about the greatest good for the greatest number \cite{ismail2018narrative}, they will still have to be careful about the amounts of non-renewable resources they use as well as the impact of their project on the environment. They will have to, considering the fact that not doing so will mean significant negative impact on the environment and the ecosystem, the planet on which all live, follow the then regulations that ensures the bettering of the damage, to minimize the damage to all the people’s ecosystem.

%%%%%%%%%%%%%%%%%%%%%%%%%%%%%%%%%%%%%%%%%%%%%%%%%%%%%%%%%%%%%%%%%%%%%%%%%%%%%%%%
\section{CONCLUSION}

The engineers will have the most restrictions in the coming decade in their productions. They will have the to contend with the moral issues of providing for the needs of the people as well as not harming the environment and the ecosystem on which the same people line in. Doing so to make sure they do not violate any of the people’s general ethical rights. This will impact more negatively the South African and developing countries engineers.

%%%%%%%%%%%%%%%%%%%%%%%%%%%%%%%%%%%%%%%%%%%%%%%%%%%%%%%%%%%%%%%%%%%%%%%%%%%%%%%%%

\nocite{*}
\bibliographystyle{witseie}
\bibliography{sample}
%{\tiny \vfill \hfill \today \hspace{5mm} witseie-paper-2003.\TeX}

\end{document}

" vim: ts=4
" vim: tw=78
" vim: autoindent
" vim: shiftwidth=4
