%%%%%%%%%%%%%%%%%%%%%%%%%%%%%%%%%%%%%%%%%%%%%%%%%%%%%%%%%%%%%%%%%%%%%%%%%%%%%%%
%
%        witseiepaper-2005.tex
%
%  Nduvho Ramashia - 30 March 2023
%
%  Sociology Essay 1
%
%%%%%%%%%%%%%%%%%%%%%%%%%%%%%%%%%%%%%%%%%%%%%%%%%%%%%%%%%%%%%%%%%%%%%%%%%%%%%%%

\documentclass[12pt]{witseiepaper} 

%
% All KJN's macros and goodies (some shameless borrowing from SPL)
%\usepackage{KJN}
\usepackage{comment}

%\bibliographystyle{witseie}
\usepackage[authoryear]{natbib}
\bibliographystyle{plainnat}

%
% PDF Info
%

\pdfinfo{
/Title (The Looming Era of Ethical Minefields: Challenges awaiting engineers in the decade ahead)
/Author (Nduvho E Ramashia)
/CreationDate (D:202303302200)
/ModDate (D:202303302200)
/Subject (ELEN4019 Sociology, 2023)
/Keywords (ELEN4019, Sociology)
}

\setlength{\hoffset}{0.05in}

%%%%%%%%%%%%%%%%%%%%%%%%%%%%%%%%%%%%%%%%%%%%%%%%%%%%%%%%%%%%%%%%%%%%%%%%%%%%%%%
\begin{document}

\title{The Looming Era of Ethical Minefields: Challenges awaiting engineers in the decade ahead}

\author{Nduvho E. Ramashia -- 1490804 
\thanks{School of Electrical \& Information Engineering, University of the Witwatersrand, Private Bag 3, 2050, Johannesburg, South Africa} 
} 

%%%%%%%%%%%%%%%%%%%%%%%%%%%%%%%%%%%%%%%%%%%%%%%%%%%%%%%%%%%%%%%%%%%%%%%%%%%%%%%%%

\abstract{N/A}
\keywords{ELEN4019, Sociology}


\maketitle
\thispagestyle{empty}
\pagestyle{empty}

%%%%%%%%%%%%%%%%%%%%%%%%%%%%%%%%%%%%%%%%%%%%%%%%%%%%%%%%%%%%%%%%%%%%%%%%%%%%%%%%

% Select a moral problem you envision to be a challenge for engineers in a decade’s time, 
% evaluating this through the lens of both Teleological and Deontological ethical frameworks.”

\section{\raggedright INTRODUCTION} 

    \subsection{The Global Perspective} 

    From its inception, the engineering profession has always been about meeting the ever expanding needs and wants of the people \citep{cruickshank2003changing}. The previous centuries' engineers brought about many vital and foundational inventions. Most of which we still use to this day \citep{cruickshank2003changing}. This include but not limited inventions such as bridges, cars, electricity and computers, to mention some. However, they never had to worry much about the amount of natural resources they were using. Neither did they have to put much consideration into the overall long-term effects their work were having on the environment, at least not as much as the engineers of today have to. Given the abundance of resources they had at their disposal then, it is only understandable. Due to the rich and favorable environment in the past, engineers didn't have to worry about the environmental impact of their designs as much as the engineers of today and the coming decade.
    
    They could not then feel the urgency to protect the environmental, not as much the engineers in the coming decade will. All they had to worry about was about meeting their engineering duties, fulfilling the needs and wants of the people. They rarely had to worry about whether or not it was moral to implement their solutions or implement any design to wanted. On the other hand, these are the challenges the engineers of the coming decade will have to cautiously navigate. There will be a need to be more conscious of the effects their work has. These effects will range from social and psychological effects, to the wider ecological effects. These being from the ever expanding online innovation environment as well as from the use of the finite physical resources at their their disposal. 
    
    Future engineers have to work with and at an already at-risk environment \citep{macnaghten2006risk}. The Intergovernmental Panel on Climate Change (IPCC) projections derived from simulations performed for the IPCC 4th Assessment Report shows how the rainfalls, amongst other things, will have worsened by the year 2030 \citep{suppiah2007australian}. These facts and predictions shines upon the pressure the future engineers will have to deal with. They show the difficulties they will have in fulfilling their duties, meeting the needs and wants of the people, while also not harming the people's environment. They will be faced with the challenge of delivering services under much tighter constraints.
    On one side they will have the need to fulfil the needs of the people, to ensure decent lives for all. On the other hand they will have to carefully make their design and choices to make sure they do not harm the ecosystem nor any socially of psychologically. 
    
    These facts show the moral challenges the future engineers will face with balancing Edward B. Barbier’s three cycles \citep{barbier1987concept}, more commonly labelled ‘economic’, ‘society’ and ‘environment’. The imbalance of the cycles will results in the violation of the ethical values autonomy, beneficence, and justice. This could be in the form of people not getting or affording descent lives, or not having their health and other needs met. It could also be in the people’s place of residence, the environment, not getting treated well, which is not good for the people as a whole. The engineers in a decade’s time will be more frequently faced with situations where they will have to consciously make sure they do not violate any of the general ethical values/rights. 

%--------------------------------------------------------------------------

    \subsection{The Impact on South Africa} 

    Given the fact that South Africa is still a developing country \citep{bakari2018south}, its engineers will be even highly impacted. Other, though not all, parts of the world have made significant technological advances back when the shared ecosystem in good health. It was also then when the world was less conscious of the environmental and ecosystem damages done by the products engineers made. They as a result got ahead technologically as well as economically. South Africa on the other, due to its complicated history and unique sets priorities, did not. Unfortunately the ecosystem harm was still done for all even then. This puts South Africa and its current and future engineers at a greater disadvantage. This being because they, now and will still have in to a decade’s time, will need to be equally careful and considerate of the ramifications of their work, irregardless of who contributed what harm amount to the then current ecosystem harm. 
    The effects will be especially dire to the countries like South Africa, countries that are not yet fully developed. This mainly because of the need for them to ensure better living for their people, the need to keep up with living standards. 
    
%%%%%%%%%%%%%%%%%%%%%%%%%%%%%%%%%%%%%%%%%%%%%%%%%%%%%%%%%%%%%%%%%%%%%%%%%%%%%%%%



\section{\raggedright Problem analysis}

Given the state the environment has been projected to be going to be in the near future \citep{collins2014long}, thinking about the overall environmental impacts the engineers make will be of even greater importance. Similar challenges will be faced in an attempt to develop and engineer internet and online tools, while still respecting the people's core ethical values. 

The values that will be at violation risk includes the autonomy ethical value. With the rise of AI and other online tools, the people's privacy violations will continue. These violations will lead to the perpetrators dictating the behaviors of others, without them even knowing. Hence it will be vital for software engineers to carefully think about the ramifications of their projects, considering all the possible moral implications. They will have to also consider the social and psychological impacts their products will be having on the people and families whole lives they are to improve. 

The engineers will have to be more careful in their designs and projects to make sure the individual corporation's asserts and valuables are not obtained at the public's expense, be it social, psychological, or environmental. They will have to try harder to make sure they do not violate the beneficence ethical people's right. 
As raw resources get fewer, the competition to provide services that involve those resources ethically gets tougher. With the large amounts of data that will be available to them by then, they will also need to be more careful to be sure they do not use then unethically for individual's benefits over the public's. They will have to do so to not only be ethical but also to keep the people's trust \citep{Masur2020, Smith2011}. 

    \subsection{Deontological View of the Challenge} 

    From the deontological point of view, the natural-resources-using engineers will have to frequently contend with the ways which they go about implementing their solutions. Closely observing whether or not it will moral or not to implement the projects faced with. This will be the case for both projects that make use of natural resources, as well as projects that will be used by the public in virtual environments.
    
    Given the fact that natural resources are finite, hence are running out \citep{subramanian2018crisis}, they will have to be more careful with their usage amounts. They will have to be sure they are not fundamentally wrong in using however much resources they need. More specifically looking at it from the Categorical Imperative point of view, which defines the morality of an action based on ramification should the act be made universally moral \citep{kant1993groundwork}.
    
    The engineers will have to stay within the bounds of natural resources uses as well as unwanted by-products, as it will be the right thing to do, as per the Categorical Imperative concept \citep{basara2018kant}, since everyone not doing so will ensure the critical damage to the ecosystem \citep{appannagari2017environmental}. The similar will be the case for virtual and software environment engineers. They will have to be consider the impact of their products on the people socially and psychologically. They will have to consider what would happen to the society if everyone gave the society the same or similar products, or used the data they same way they may be tempted to. Considering carefully the impacts on the society.  
    
    On the other hand, they will still have to provide for the needs of the people, as well as to grow and develop the country, for still developing countries like South Africa. They will have to, on every project or innovation, battle with the morality of the project due to the limited resources and the ever dying ecosystem, as well the social and psychological impacts on the society. They will have to do so more frequently since they are one of the main professions that work with and use the environment and whose products contribute to the bad health of the ecosystem. They will also be the people in position to make innovations that may if not properly and carefully implemented have critical impacts on the social and psychological impacts on the people. On top of the impacts on individual people's psyches and groups, communities and cultures' impacts, they are responsible for creating and putting tools on people's hands. This comes with its ethical responsibilities in itself. With the increasing capabilities of AI and software tools in general, engineers will more frequently faced with moral dilemmas where they have to think carefully about the morality of creating some tools. This being because some some tools may have the potential to save individual people's money at other people's jobs and such. The increased potential of technology by then will make these considerations more complex that they are today, considering what would happen if everyone started implementing similar solutions and tools.   
 %-------------------------------------------------------------------------- 

    \subsection{Teleological Point of View} 

    Looking at it from the teleological point of view, which focuses more on the consequences of an action \citep{ismail2018narrative}, they will still have to be careful, or else they end up bringing about the irreversible damage of the ecosystem, which is not a wanted or acceptable outcome or consequence by many forms of teleological frameworks. 
    Looking at it more specifically from the Utilitarian point of view, whose approach is about the greatest good for the greatest number \citep{ismail2018narrative, ethicsunwrapped}, they will still have to be careful about the amounts of non-renewable resources they use as well as the impact of their project on the environment. They will have to, considering the fact that not doing so will mean significant negative impact on the environment and the ecosystem, the planet on which all live, follow the then regulations that ensures the bettering of the damage, to minimize the damage to all the people’s ecosystem. 

    They will also, from a utilitarian point of view, carefully consider the impacts of their tools and services on the overall society's social and psychological impacts, hence they will be faced with the similar difficulties as the dermatologists. 
    

%%%%%%%%%%%%%%%%%%%%%%%%%%%%%%%%%%%%%%%%%%%%%%%%%%%%%%%%%%%%%%%%%%%%%%%%%%%%%%%%



\section{\raggedright\setlength{\hoffset}{0.1in} CONCLUSION}

The engineers will have the most restrictions in the coming decade in their productions. These restrictions will come from the resultant ramifications of both the use of the limited natural resources, as well as the as what creating some products would do the society's social and psychological well-being. They will more frequently faced with dilemmas where they have to fulfil their duties of providing tools and services to improve people's lives, but also ensuring they are doing so. They will more constraints that people today and in the past have and have had. 
They will have to do these at all times to make sure they do not violate any of the people’s core ethical rights. The challenges will be there whether one looks at it from the deontological or teleological point of view. 
This will impact more severely the engineer in developing countries, such as South African.

%%%%%%%%%%%%%%%%%%%%%%%%%%%%%%%%%%%%%%%%%%%%%%%%%%%%%%%%%%%%%%%%%%%%%%%%%%%%%%%%%

\nocite{*}
\bibliography{sample}
%{\tiny \vfill \hfill \today \hspace{5mm} witseie-paper-2003.\TeX}

\end{document}

" vim: ts=4
" vim: tw=78
" vim: autoindent
" vim: shiftwidth=4
